\chapter{Lezione 2 - 26/09}



\section{Polinomi}

\Definizione{Polinomi}{
Sia \( (K, +, \cdot) \) un campo e \( x \) una variabile (detta anche \emph{incognita}).

\begin{itemize}
    \item Per ogni \( n \in \mathbb{N} \cup \{0\} \) si definisce \( x^n = \underbrace{x \cdot x \cdot \ldots \cdot x}_{n \text{ volte}} \), con la convenzione \(x^0 = 1\).
    \item Un \textbf{polinomio} nella variabile \(x\) a coefficienti in \(K\) è una somma finita di potenze di \(x\) moltiplicate per scalari in \(K\):
    \[
        p(x) = a_0 + a_1 x + a_2 x^2 + \ldots + a_d x^d = \sum_{i=0}^{d} a_i x^i,
    \]
    dove \( a_0, a_1, \ldots, a_d \in K \) e \( a_d \neq 0 \).
\end{itemize}

Il numero naturale \(d\) è detto \textbf{grado} del polinomio, e si indica con
\[
    \mathrm{gr}(p) = \max\{ i \mid a_i \neq 0 \}.
\]

L’insieme di tutti i polinomi in una variabile \(x\) a coefficienti in \(K\) si indica con
\[
    K[x] = \{ p(x) \mid p(x) \text{ è un polinomio in } x \text{ con coefficienti in } K \}.
\]
}

---

\BoxBlue{Osservazione:}{Operazioni in \(K[x]\)}{
In \(K[x]\) sono definite naturalmente le seguenti operazioni:

\begin{itemize}
    \item \textbf{Somma di polinomi:}
    \[
        (p+q)(x) = \sum_{i=0}^{\max(d_p, d_q)} (a_i + b_i) x^i, \quad p(x) = \sum a_i x^i, \; q(x) = \sum b_i x^i.
    \]
    Esempio:
    \[
        (2x^2 + 3x + 1) + (x^2 - x + 4) = 3x^2 + 2x + 5.
    \]

    \item \textbf{Moltiplicazione per scalare:}
    \[
        (\lambda p)(x) = \sum_{i=0}^{d} (\lambda a_i) x^i, \quad \lambda \in K.
    \]
    Esempio:
    \[
        3(2x^2 + x + 1) = 6x^2 + 3x + 3.
    \]

    \item \textbf{Prodotto di polinomi:}
    \[
        (p \cdot q)(x) = \left( \sum_{i=0}^{d} a_i x^i \right)\left( \sum_{\alpha=0}^{e} b_\alpha x^\alpha \right)
        = \sum_{k=0}^{d+e} \left( \sum_{i+\alpha = k} a_i b_\alpha \right) x^k.
    \]
    Esempio:
    \[
        (x+1)(x^2+x+1) = x^3 + 2x^2 + 2x + 1.
    \]
\end{itemize}

Con queste operazioni \( (K[x], +, \cdot) \) è un \textbf{anello commutativo con identità} e, rispetto alla sola somma, uno \textbf{spazio vettoriale} su \(K\).
}

---

\Definizione{Polinomi in più variabili}{
Siano \(x_1, x_2, \ldots, x_n\) variabili e \(K\) un campo.

Un \textbf{monomio} nelle variabili \(x_1, \ldots, x_n\) è un prodotto del tipo
\[
    x^\alpha = x_1^{\alpha_1} x_2^{\alpha_2} \cdots x_n^{\alpha_n},
\]
dove \(\alpha = (\alpha_1, \ldots, \alpha_n) \in \mathbb{N}^n\).

Un \textbf{polinomio in \(n\) variabili} a coefficienti in \(K\) è una somma finita di monomi moltiplicati per scalari:
\[
    p(x_1, \ldots, x_n) = \sum_{\alpha \in A} a_\alpha x^\alpha, \quad a_\alpha \in K,
\]
dove \(A \subset \mathbb{N}^n\) è un insieme finito.

L’insieme di tutti i polinomi in \(n\) variabili a coefficienti in \(K\) si indica con
\[
    K[x_1, \ldots, x_n].
\]
}

---

\Definizione{Grado di un polinomio in più variabili}{
Sia \(p(x_1,\ldots,x_n) = \sum_\alpha a_\alpha x^\alpha\), con \(a_\alpha \neq 0\) per un certo numero finito di \(\alpha\).

\begin{itemize}
    \item Il \textbf{grado del monomio} \(x^\alpha = x_1^{\alpha_1}\cdots x_n^{\alpha_n}\) è
    \[
        \mathrm{gr}(x^\alpha) = \sum_{i=1}^n \alpha_i.
    \]
    \item Il \textbf{grado del polinomio} è
    \[
        \mathrm{gr}(p) = \max\{ \mathrm{gr}(x^\alpha) \mid a_\alpha \neq 0 \}.
    \]
\end{itemize}

Esempio:
\[
p(x,y,z) = 2x^2y + 3xyz^3 - 5z^2 \quad \Rightarrow \quad \mathrm{gr}(p) = \max\{3,5,2\} = 5.
\]
}

\Definizione{Polinomio lineare}{
Un \textbf{polinomio lineare} in una variabile \(x\) su un campo \(K\) è un polinomio del tipo
\[
    p(x) = a_1 x + a_0,
\]
dove \(a_1, a_0 \in K\) e \(a_1 \neq 0\).

\begin{itemize}
    \item Il \textbf{grado} di un polinomio lineare è \(1\), poiché il termine di grado più alto è \(a_1 x\).
    \item Il termine \(a_1\) si chiama \textbf{coefficiente angolare} o \textbf{coefficiente direttore}.
    \item Il termine \(a_0\) si chiama \textbf{termine noto}.
\end{itemize}

Nel caso di più variabili, un \textbf{polinomio lineare} in \(x_1, x_2, \ldots, x_n\) è della forma
\[
    p(x_1, x_2, \ldots, x_n) = a_1 x_1 + a_2 x_2 + \ldots + a_n x_n + a_0,
\]
con \(a_0, a_1, \ldots, a_n \in K\) e almeno uno tra \(a_1, \ldots, a_n\) non nullo.
}

\section{Matrici}

\Definizione{Matrice}{
Siano \(m, n \in \mathbb{N}\) e sia \(X\) un insieme non vuoto.  
Una \textbf{matrice di tipo} \(m \times n\) a valori in \(X\) è una applicazione
\[
    A : \{1, 2, \ldots, m\} \times \{1, 2, \ldots, n\} \longrightarrow X
\]
che associa a ogni coppia \((i, j)\) un elemento \(a_{ij} \in X\).

Scriviamo
\[
    A = (a_{ij})_{m \times n} =
    \begin{pmatrix}
        a_{11} & a_{12} & \cdots & a_{1n} \\
        a_{21} & a_{22} & \cdots & a_{2n} \\
        \vdots & \vdots & \ddots & \vdots \\
        a_{m1} & a_{m2} & \cdots & a_{mn}
    \end{pmatrix}.
\]

L’insieme di tutte le matrici \(m \times n\) a valori in \(X\) si indica con \(M_{m,n}(X)\).
}

\Esempio{Matrice \(3 \times 2\) su un insieme \(X\)}{
Siano
\[
    m = 3, \quad n = 2, \quad X = \{\pi, \sqrt{2}, \square, \star, y\}.
\]
Definiamo
\[
    A : \{1,2,3\} \times \{1,2\} \longrightarrow X
\]
tale che:
\[
\begin{aligned}
A(1,1) &= \pi, &\quad A(1,2) &= \sqrt{2},\\
A(2,1) &= \square, &\quad A(2,2) &= \star,\\
A(3,1) &= y, &\quad A(3,2) &= \pi.
\end{aligned}
\]
Allora la matrice \(A\) è:
\[
A =
\begin{pmatrix}
    \pi & \sqrt{2} \\
    \square & \star \\
    y & \pi
\end{pmatrix}.
\]
}

\BoxBlue{Osservazioni:}{Operazioni sulle matrici}{
    Siano \(A, B \in M_{m,n}(K)\).  
    \begin{itemize}
        \item La loro somma è la matrice \(C = A + B\) definita da:
        \[
            c_{ij} = a_{ij} + b_{ij}, \quad \forall i=1,\ldots,m, \; j=1,\ldots,n.
        \]
    
        \Esempio{Somma di matrici}{
        \[
        A =
        \begin{pmatrix}
        1 & 2 \\
        3 & 4
        \end{pmatrix},
        \quad
        B =
        \begin{pmatrix}
        5 & 6 \\
        7 & 8
        \end{pmatrix}
        \Rightarrow
        A + B =
        \begin{pmatrix}
        6 & 8 \\
        10 & 12
        \end{pmatrix}.
        \]
        }
        \item Sia \(A = (a_{ij}) \in M_{m,n}(K)\) e \(\lambda \in K\).  
        Definiamo la matrice \(\lambda A = (\lambda a_{ij})\), cioè:
        \[
            (\lambda A)_{ij} = \lambda \cdot a_{ij}.
        \]
    
        \Esempio{Moltiplicazione per uno scalare}{
        \[
        \lambda = 3, \quad
        A =
        \begin{pmatrix}
        1 & -2 \\
        0 & 4
        \end{pmatrix}
        \Rightarrow
        3A =
        \begin{pmatrix}
        3 & -6 \\
        0 & 12
        \end{pmatrix}.
        \]
        }
    \end{itemize}

}