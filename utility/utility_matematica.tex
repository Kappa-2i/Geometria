% librerie matematica
\usepackage{siunitx} % fornisce dei typeset per i numeri e le unita di misura
\usepackage{amssymb}
\usepackage{amsthm}

% librerie geometria
\usepackage{mathtools}   % estende amsmath, utile per simboli e allineamenti
\usepackage{bm}          % bold math (per vettori e simboli)
\usepackage{physics}     % comandi rapidi per derivate, vettori, matrici
\usepackage{tikz}        % disegni geometrici
\usepackage{tikz-cd}     % diagrammi commutativi


\usepackage{amsfonts}

% emoji
%\usepackage{fourier}
\usepackage{emoji}

% creazione dei Toast message
\usepackage[many]{tcolorbox}

% definizione di nuovi colori
\usepackage{xcolor}

% SIMBOLO MATEMATICO ESISTE ED E' UNICO
\newcommand{\existsone}{\exists !}

% INSIEMI MATEMATICI
\newcommand{\R}{\mathbb{R}}
\newcommand{\N}{\mathbb{N}}
\newcommand{\Z}{\mathbb{Z}}
\newcommand{\C}{\mathbb{C}}

% DEFINIZIONE
\definecolor{myred}{RGB}{191,101,127}
\definecolor{mygreen}{RGB}{0,168,107}

\definecolor{myblue}{RGB}{0,127,255}
\definecolor{myblue2}{RGB}{36,84,255}

\definecolor{myyellow}{RGB}{255,191,0}
\definecolor{mygray}{RGB}{83,104,120}

% BOX VUOTI
\newtcolorbox{empty_green}{enhanced, toprule=0pt, bottomrule=0pt, rightrule=0pt, leftrule=3pt, arc=0mm, skin=enhancedlast jigsaw, sharp corners,
colframe=mygreen, colbacktitle=mygreen}

\newtcolorbox{empty_red}{breakable, enhanced, toprule=0pt, bottomrule=0pt, rightrule=0pt, leftrule=3pt, arc=0mm, skin=enhancedlast jigsaw, sharp corners,
colframe=myred, colbacktitle=myred}

\newtcolorbox{empty_blue}{enhanced, toprule=0pt, bottomrule=0pt, rightrule=0pt, leftrule=3pt, arc=0mm, skin=enhancedlast jigsaw, sharp corners,
colframe=myblue, colbacktitle=myblue}

\newtcolorbox{empty_yellow}{enhanced, toprule=0pt, bottomrule=0pt, rightrule=0pt, leftrule=3pt, arc=0mm, skin=enhancedlast jigsaw, sharp corners,
colframe=myyellow, colbacktitle=myyellow}


\newcommand{\BoxRed}[3]{
    \begin{empty_red}
        \textcolor{myred}{\textbf{#1} \textbf{#2}}\\
        \noindent #3
    \end{empty_red}
}

\newcommand{\BoxGreen}[3]{
    \begin{empty_green}
        \textcolor{mygreen}{\textbf{#1} \textbf{#2}}\\
        \noindent #3
    \end{empty_green}
}

\newcommand{\BoxBlue}[3]{
    \begin{empty_blue}
        \textcolor{myblue}{\textbf{#1} \textbf{#2}}\\
        \noindent #3
    \end{empty_blue}
}

\newcommand{\Attenzione}[1]{
    \begin{empty_yellow}
        \emoji{construction} #1
    \end{empty_yellow}
}

\newcommand{\Definizione}[2]{
    \begin{empty_green}
         \textcolor{mygreen}{\textbf{Definizione (#1)}}\\
        \noindent #2
    \end{empty_green}
}

\newcommand{\Teorema}[2]{
    \begin{empty_red}
        \textcolor{myred}{\textbf{Teorema (#1)}}\\
        \noindent #2
    \end{empty_red}
}

\newcommand{\Dimostrazione}[1]{
    \begin{empty_blue}
        \textcolor{myblue}{\textbf{Dimostrazione}}\\
        \noindent #1
    \end{empty_blue}
}

% vettore in grassetto (più leggibile di \vec)
\newcommand{\vect}[1]{\mathbf{#1}}

% alternativa: freccia sopra
\newcommand{\vett}[1]{\vec{#1}}

% matrice generica
\newcommand{\mat}[1]{\begin{bmatrix} #1 \end{bmatrix}}

% matrice piccola inline
\newcommand{\smat}[1]{\begin{smallmatrix} #1 \end{smallmatrix}}

\newcommand{\scal}[2]{\left\langle #1, #2 \right\rangle}
\newcommand{\norma}[1]{\left\lVert #1 \right\rVert}

% Sistema lineare generico
\newcommand{\SistemaGenerale}{
    \begin{cases}
        a_{11}x_1 + a_{12}x_2 + \dots + a_{1n}x_n - b_1 = 0 \\
        a_{21}x_1 + a_{22}x_2 + \dots + a_{2n}x_n - b_2 = 0 \\
        \quad\vdots \\
        a_{m1}x_1 + a_{m2}x_2 + \dots + a_{mn}x_n - b_m = 0
    \end{cases}
}

\newtcolorbox{ProposizioneBox}[2][]{
  enhanced,
  colback=white,
  colframe=myblue,
  coltitle=black,
  fonttitle=\bfseries,
  title={\textbf{Proposizione:} #2},
  attach boxed title to top left={xshift=0.5em,yshift=-2mm},
  boxed title style={
    colback=myblue!10,
    colframe=myblue,
    rounded corners,
    top=2pt,
    bottom=2pt,
    left=5pt,
    right=5pt
  },
  sharp corners,
  boxrule=0.5pt,
  rounded corners,
  drop shadow,
  breakable,
  before skip=10pt,
  after skip=10pt
}

% Comando semplificato per chiamare l'ambiente
\newcommand{\Proposizione}[2]{
  \begin{ProposizioneBox}{#1}
    #2
  \end{ProposizioneBox}
}



