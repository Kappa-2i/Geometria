\chapter{Lezione 4 - 3/10 (Teams)}

\Teorema{Algoritmo di Gauss}{
Ogni matrice su un campo \(K\) può essere ridotta o completamente ridotta a gradini mediante un numero finito di operazioni elementari.

\vspace{1em} % Spazio prima della dimostrazione

\textcolor{myred}{\textbf{Dimostrazione (per induzione)}}\\[0.5em]
Procediamo per induzione sul numero di righe \(m\):

\begin{itemize}
    \item \textbf{Per \(m = 1\):}  
    \[
        A = (a_{11}, a_{12}, \dots, a_{1n})
    \]
    Se \(a_{1j}\) è il pivot, basta eseguire la seguente operazione di normalizzazione (ridurre a 1):  
    \[
        a^{(1)} \;\to\; \frac{1}{a_{1j}}\, a^{(1)}.
    \]

    \item \textbf{Per \(m > 1\):}  
    Dimostriamo che, se vale per \(m - 1\), allora vale anche per \(m\).  
    Prima di tutto, individuiamo la prima riga non nulla:
    \[
        A =
        \begin{bmatrix}
            a_{11} & a_{1j} & \dots & a_{1n} \\
            a_{21} & a_{2j} & \dots & a_{2n} \\
            \vdots & \vdots & \ddots & \vdots \\
            a_{m1} & a_{mj} & \dots & a_{mn}
        \end{bmatrix}.
    \]

    Sia
    \[
        j = \min\{\, l \in \{1, \dots, n\} \mid a_l \neq 0 \,\}
    \]
    la posizione del primo elemento non nullo nella prima riga.

    Analogamente, poniamo
    \[
        k = \min\{\, i \in \{1, \dots, m\} \mid a_{ij} \neq 0 \,\},
    \]
    cioè l'indice della prima riga (a partire dall'alto) che contiene un elemento non nullo nella colonna \(j\).

    Una volta individuati tali indici, scambiamo la riga \(a_k\) con la prima riga \(a_1\):
    \[
        a_k \;\longleftrightarrow\; a_1.
    \]

    Dopo lo scambio, otteniamo:
    \[
        A' =
        \begin{bmatrix}
            a'_{11} & a'_{1j} & \dots & a'_{1n} \\
            a'_{21} & a'_{2j} & \dots & a'_{2n} \\
            \vdots  & \vdots  & \ddots & \vdots \\
            a'_{m1} & a'_{mj} & \dots & a'_{mn}
        \end{bmatrix}.
    \]

    Dato \(a'_{1j}\) (pivot della prima riga, diverso da 0), dobbiamo annullare tutti gli elementi sottostanti.  
    Per la riga 2, dobbiamo avere:
    \[
        a'_{2J} + B a'_{1J} = 0.
    \]
    Basta prendere \(B = -\dfrac{a'_{2J}}{a'_{1J}}\).

    Procediamo quindi in questo modo per tutte le righe sottostanti, ottenendo:
    \[
    \begin{aligned}
        a^{(2)} &\;\to\; a^{(2)} - \frac{a'_{2J}}{a'_{1J}}\, a^{(1)}, \\
        a^{(3)} &\;\to\; a^{(3)} - \frac{a'_{3J}}{a'_{1J}}\, a^{(1)}, \\
        &\;\;\vdots \\
        a^{(m)} &\;\to\; a^{(m)} - \frac{a'_{mJ}}{a'_{1J}}\, a^{(1)}.
    \end{aligned}
    \]


    Se si vuole la matrice completamente ridotta, bisogna normalizzare il pivot \(a''_{kj}\) di ogni riga e annullare gli elementi della
    stessa riga che si trovano sopra ai pivot delle righe successive. 

\end{itemize}
                

}

\BoxBlue{Osservazione:}{Derivata dal Teorema di Rouchè-Capelli}{
Un sistema lineare \Sigma: Ax=b è compatibile \Leftrightarrow la matrice completa una volta ridotta 
a gradini, tramite operazioni elementari, non presenta una riga in cui tutti gli elementi della matrice
dei coefficienti sono 0 ma l'elemento nella colonna dei termini noti è diverso da 0. 

\[
\Sigma : 
\begin{cases}
x_1 + x_2 = 1 \\[2mm]
2x_1 + 2x_2 = 3
\end{cases}
\]

\[
\text{Matrice completa:} \quad
C =
\begin{bmatrix}
1 & 1 & | & 1 \\
2 & 2 & | & 3
\end{bmatrix}
\]

\[
\text{Applichiamo } R_2 \to R_2 - 2 R_1:
\]

\[
C =
\begin{bmatrix}
1 & 1 & | & 1 \\
0 & 0 & | & 1
\end{bmatrix}
\]

\[
\text{ il sistema è impossibile.}
\]


} 

\Teorema{Teorema di struttura dell'insieme delle soluzioni di un sistema lineare}{
Sia 
\(\Sigma : A \mathbf{x} = \mathbf{b}\) un sistema lineare di equazioni in \(n\) incognite e sia 
\(\Sigma_0: A \mathbf{x} = \mathbf{0}\)
il sistema lineare omogeneo associato.

Sia 
\[
\mathbf{S} = \{ (y_1, \dots, y_n) \in K^n \mid A 
\begin{pmatrix} y_1 \\ \vdots \\ y_n \end{pmatrix} = \mathbf{b} \}
\] 
l'insieme delle soluzioni di \(\Sigma\), e sia 
\[
\mathbf{S}_0 = \{ (z_1, \dots, z_n) \in K^n \mid A 
\begin{pmatrix} z_1 \\ \vdots \\ z_n \end{pmatrix} = \mathbf{0} \}
\] 
l'insieme delle soluzioni di \(\Sigma_0\).

Allora, se \((\bar{y}_1, \dots, \bar{y}_n) \in \mathbf{S}\), si ha
\[
\mathbf{S} = \{ (\bar{y}_1, \dots, \bar{y}_n) + (z_1, \dots, z_n) \mid (z_1, \dots, z_n) \in \mathbf{S}_0 \} = \mathbf{X} .
\]



\vspace{0.5em}
\textcolor{myred}{\textbf{Dimostrazione:}}\\[0.5em]
Mostriamo che \(\mathbf{S} \subseteq \mathbf{X}\) verificando le due inclusioni:

\begin{enumerate}
    \item \(\mathbf{S} \subseteq \mathbf{X}\) :  
    Sia \((y_1, \dots, y_n)\) \in \mathbf{S}.  
    Consideriamo \((z_1, \dots, z_n) = (y_1, \dots, y_n) - (\bar{y}_1, \dots, \bar{y}_n)\).  
    Poiché 
    \[
        A\mathbf{z} = A(\mathbf{y} - \bar{\mathbf{y}}) = A\mathbf{y} - A\bar{\mathbf{y}} = \mathbf{b} - \mathbf{b} = \mathbf{0},
    \]
    si ha \(\mathbf{z} \in \mathbf{S}_0\). Quindi \(\mathbf{y} = \bar{\mathbf{y}} + \mathbf{z} \in \mathbf{X}\).

    \item \(\mathbf{X} \subseteq \mathbf{S}\) :  
    Sia \(\mathbf{x} \in \mathbf{X}\). Allora \(\mathbf{x} = \bar{\mathbf{y}} + \mathbf{z}\) con \(\mathbf{z} \in \mathbf{S}_0\).  
    Allora
    \[
        A\mathbf{x} = A(\bar{\mathbf{y}} + \mathbf{z}) = A\bar{\mathbf{y}} + A\mathbf{z} = \mathbf{b} + \mathbf{0} = \mathbf{b},
    \]
    quindi \(\mathbf{x} \in \mathbf{S}\).
\end{enumerate}

Da queste due inclusioni segue che \(\mathbf{S} = \mathbf{X}\). \(\quad \square\)
}





