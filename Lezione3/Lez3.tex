\chapter{Lezione 3 - 01/10/2025}

\section{Matrice Trasposta}
\Definizione{Matrice Trasposta}{
    Sia \(A \in M_{m,n}(K)\). La trasposta di \(A\), denotata \({}^{t}A\), è la matrice del tipo \([n,m]\) che come righe ha le colonne di A.\newline\newline
    \({}^{t}A\)=\(\mat{
        a_{11} & a_{21} & \dots & a_{m1} \\
        a_{12} & a_{22} & \dots & a_{m2} \\
        \vdots & \vdots & \ddots & \vdots \\
        a_{1n} & a_{2n} & \dots & a_{mn}
    } \in M_{n,m}(K)\), ottenuta scambiando righe e colonne di \(A\).
}

\section{Prodotto Scalare}
\Definizione{Prodotto scalare}{
Sia \(K = \mathbb{R}\) e siano due vettori 
\(\vect{a} = (a_1, a_2, \dots, a_n), \vect{b} = (b_1, b_2, \dots, b_n) \in K^n\).  

Il \textbf{prodotto scalare} è la funzione
\[
K^n \times K^n \longrightarrow K, \quad (\vect{a}, \vect{b}) \longmapsto a_1 b_1 + a_2 b_2 + \dots + a_n b_n = \scal{\vect{a}}{\vect{b}},
\]
che associa la coppia di vettori ad uno scalare, dato dalla somma delle componenti omonime dei due vettori.
}

\section{Matrice Conformabile}
\Definizione{Matrice Conformabile}{
    Due matrici \(A \in M_{m,n}(K)\) e \(B \in M_{p,q}(K)\) si dicono \textbf{conformabili} per il prodotto se e solo se il numero di colonne di \(A\) è uguale al numero di righe di \(B\), cioè \(n = p\). In tal caso, il prodotto \(AB\) è definito ed è una matrice di dimensione \(m \times q\).

    \textbf{Esempio:}  
    Siano
    \[
        A = \begin{bmatrix} 1 & 2 \\ 3 & 4 \\ 5 & 6 \end{bmatrix} \in M_{3,2}(K), 
        \quad
        B = \begin{bmatrix} 7 & 8 & 9 \\ 10 & 11 & 12 \end{bmatrix} \in M_{2,3}(K).
    \]
    Allora \(A\) e \(B\) sono conformabili e
    \[
        AB = \begin{bmatrix} 1\cdot7+2\cdot10 & 1\cdot8+2\cdot11 & 1\cdot9+2\cdot12 \\
                             3\cdot7+4\cdot10 & 3\cdot8+4\cdot11 & 3\cdot9+4\cdot12 \\
                             5\cdot7+6\cdot10 & 5\cdot8+6\cdot11 & 5\cdot9+6\cdot12
              \end{bmatrix} \in M_{3,3}(K).
    \]
}

\section{Prodotto Riga per Colonna}
\Definizione{Prodotto Riga Per Colonna}{
    Siano \(A \in M_{m,n}(K)\) e \(B \in M_{n,p}(K)\). Il prodotto di una riga \(i\)-esima di \(A\) per una colonna \(j\)-esima di \(B\) è definito come la somma dei prodotti delle componenti corrispondenti:
    \[
        (AB)_{ij} = \sum_{k=1}^{n} A_{ik} B_{kj}.
    \]
    
    In altre parole, per ottenere l'elemento in posizione \((i,j)\) del prodotto \(AB\), si moltiplicano elemento per elemento la riga \(i\) di \(A\) con la colonna \(j\) di \(B\) e si sommano i risultati.

    \textbf{Esempio:}  
    Siano
    \[
        A = \begin{bmatrix} 1 & 2 & 3 \\ 4 & 5 & 6 \end{bmatrix}, \quad
        B = \begin{bmatrix} 7 & 8 \\ 9 & 10 \\ 11 & 12 \end{bmatrix}.
    \]
    Allora
    \[
        (AB)_{11} = 1\cdot7 + 2\cdot9 + 3\cdot11 = 58, \quad
        (AB)_{12} = 1\cdot8 + 2\cdot10 + 3\cdot12 = 68,
    \]
    e così via per gli altri elementi del prodotto.
}

\section{Sistema Lineare}
\Definizione{Sistema Lineare}{
    Siano \(m, n \in \N\), e sia \(K\) un campo. Un \textbf{sistema lineare} di \(m\) equazioni in \(n\) incognite \(x_1, x_2, \dots, x_n\) su un campo \(K\)\ è un insieme di \(m\) equazioni del tipo:
    \[
    \SistemaGenerale
    \]
    \newline
    In forma compatta, un sistema lineare può essere scritto come:
    \[
        A \vect{x} = \vect{b},
    \]
    dove:
    \begin{itemize}
        \item \(A \in M_{m,n}(K)\) è la \textbf{matrice dei coefficienti};
        \item \(\vect{x} \in K^n\) è il \textbf{vettore incognite};
        \item \(\vect{b} \in K^m\) è il \textbf{vettore dei termini noti}.
    \end{itemize}
}

\BoxBlue{Osservazione:}{Soluzioni di un sistema lineare}{
L’insieme delle soluzioni di un sistema lineare
è dato da
\[
    S_1 \text{ soluzioni della prima equazione } E_1(x)
\]
\[
    S_2 \text{ soluzioni della seconda equazione } E_2(x)
\]
\[
    \dots
\]
\[
    S_m \text{ soluzioni della m-esima equazione } E_m(x)
\]

Noi siamo interessati a 
\[
    S = \bigcap_{i=1}^m S_i,
\]
ovvero l’intersezione di tutte le soluzioni del sistema.
}

\Definizione{Compatibilità di un sistema lineare}{
Un sistema lineare \(\mathcal{E}\) si dice \textbf{compatibile} se ammette almeno una soluzione,  
ossia se l’insieme delle soluzioni è diverso dal vuoto. \newline

Altrimenti se \(S = \varnothing\), allora il sistema si dice \textbf{incompatibile}.
}

\Definizione{Matrice completa e incompleta}{
Dato un sistema lineare \(\mathcal{E}\) si distinguono due matrici:

\begin{itemize}
    \item La \textbf{matrice incompleta} (o matrice dei coefficienti) di un sistema lineare contiene solo i coefficienti delle incognite.
    \[
        A = 
        \begin{bmatrix}
            a_{11} & a_{12} & \dots & a_{1n} \\
            a_{21} & a_{22} & \dots & a_{2n} \\
            \vdots & \vdots & \ddots & \vdots\\
            a_{m1} & a_{m2} & \dots & a_{mn}
        \end{bmatrix}
    \] \newline

    \item La \textbf{matrice completa} (o matrice dei coefficienti estesa) si ottiene aggiungendo a quella incompleta una colonna aggiuntiva con i termini noti del sistema.
    \[
        C=(A \,|\, \vect{b}) = 
        \begin{bmatrix}
            a_{11} & a_{12} & \dots & a_{1n} & b_1 \\
            a_{21} & a_{22} & \dots & a_{2n} & b_2 \\
            \vdots & \vdots & \ddots & \vdots & \vdots \\
            a_{m1} & a_{m2} & \dots & a_{mn} & b_m
        \end{bmatrix}
    \]
    
\end{itemize}
}


\BoxBlue{}{Esempi di matrici complete e incomplete}{
Consideriamo due sistemi lineari in due incognite \(x_1, x_2\).

\begin{itemize}
    \item \textbf{Sistema omogeneo} \(\mathcal{E}_0\) (con termini noti nulli):
    \[
        \mathcal{E}_0 : 
        \begin{cases}
            3x_2 - x_2 + 2x_4 = 0 \\[6pt]
            x_1 +x_2 + 2x_3 = 0
        \end{cases}
    \]

    - Matrice incompleta:
    \[
        A = \mat{0 & 3 & -1 & 2 \\ 1 & 1 & 2 & 0}
    \]

    \item \textbf{Sistema non omogeneo} \(\mathcal{E}\) (con termini noti \(\neq 0\)):
    \[
        \mathcal{E} : 
        \begin{cases}
            3x_2 - x_2 + 2x_4 - 3 = 0 \\[6pt]
            x_1 +x_2 + 2x_3 + 5 = 0
        \end{cases}
    \]

    - Matrice completa:
    \[
        A = \mat{0 & 3 & -1 & 2 & | & -3\\ 1 & 1 & 2 & 0 & | & 5}
    \]
\end{itemize}
}

\section{Operazioni o Trasformazioni Elementari sulle Righe di una Matrice}

\Definizione{Operazioni elementari sulle righe}{
    Sia \(A \in M_{m,n}(K)\) una matrice.  
    Si chiamano \textbf{operazioni elementari sulle righe} le seguenti trasformazioni che possono essere applicate alle righe di \(A\):

    \begin{enumerate}
        \item \textbf{Scambio di due righe:}  
        per ogni \(h, k \in \{1, \dots, m\}\),
        \[
            \vect{b}^{(h)} \longleftrightarrow \vect{b}^{(k)}.
        \]
        
        \item \textbf{Moltiplicazione di una riga per uno scalare non nullo:}  
        per ogni \(h \in \{1, \dots, m\}\) e per ogni \(\lambda \in K \setminus \{0\}\),
        \[
            \vect{b}^{(h)} \longrightarrow \lambda \vect{b}^{(h)}.
        \]
        
        \item \textbf{Somma di una riga con un multiplo di un’altra:}  
        per ogni \(h, k \in \{1, \dots, m\}\) e per ogni \(\beta \in K\),
        \[
            \vect{b}^{(h)} \longrightarrow \vect{b}^{(h)} + \beta \vect{b}^{(k)}.
        \]
    \end{enumerate}
}

\BoxBlue{Osservazione:}{Invarianza dell’insieme delle soluzioni}{
    Le operazioni elementari sulle righe di una matrice (e quindi sul sistema lineare associato)
    \textbf{modificano la forma del sistema}, ma \textbf{non alterano il suo insieme delle soluzioni}.  
    \newline
}


