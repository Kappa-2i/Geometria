\chapter{Lezione 5 - 08/10}

\section{Sistemi di Generatori}

\Definizione{Combinazione Lineare}{
    Dati almeno due vettori \((u_1,...,u_t) \in V\) e degli scalari \((\alpha_1,...,\alpha_t) \in K\),
    la \textbf{combinazione lineare} dei vettori dati mediante gli scalari dati è il vettore seguente:
    \[\mathbf{v} = \alpha_1 u_1 + ... + \alpha_t u_t\]
    Se vogliamo permettere che ci siano ripetizioni tra i vettori e gli scalari, allora consideriamo t-uple
    \((u_1,...,u_t) \in V^t\), \((\alpha_1,...,\alpha_t) \in K^t\). 
}

\Definizione{Chiusura Lineare}{
    Dato un sottinsieme \(X \neq \emptyset\) di \(V\), diremo \textbf{chiusura lineare} di \(X\)
    il sottoinsieme \(L(X)\) di \(V\), costituito da tutti e soli i vettori che sono combinazioni lineari di vettori di \(X\).
    Se:
    \begin{itemize}
        \item \(X = \emptyset\), porremo convenzionalmente \(L(\emptyset) = {0}\)
        \item \(X={x_1,...,x_t}\) è un insieme finito, scriveremo \(L(x_1,..,x_t)\).
    \end{itemize}

    Esempio: \(L((2,3),(1,1)) = \{\alpha_1(2,3) + \alpha_2(1,1) | \alpha_1, \alpha_2 \in \mathbb{R}\} \).
}

\Definizione{Sistema di Generatori}{
    Un sistema di generatori di \(V\) è un sottinsieme di \(V\) tale che ogni vettore di \(S\) è combinazione lineare di vettori diversi, ossia \(V=L(S)\).
    Inoltre se lo spazio vettoriale \(V\) ammette un sistema di generatori finito, allora si dirà \textbf{finitamente generato}. 
}

\Proposizione{Importante sulla chiusura lineare}{
    Sia \( X \subseteq V \). Allora valgono le seguenti proprietà:
    \begin{itemize}
        \item \( X \subseteq L(X) \);
        \item \( L(X) \) è un sottospazio vettoriale di \( V \);
        \item Se \( W \subseteq V \) è un sottospazio vettoriale di \( V \) tale che \( X \subseteq W \), allora \( L(X) \subseteq W \).
    \end{itemize}

    \begin{proof}[Dimostrazione del primo punto]
        Consideriamo un qualunque \( u \in X \).  
        Poiché \( u = 1 \cdot u \), ed \( 1 \in K \), si ha che \( u \) è combinazione lineare di un elemento di \( X \).  
        Dunque \( u \in L(X) \), e quindi:
        \[
        X \subseteq L(X).
        \]
    \end{proof}

    \begin{proof}[Dimostrazione del secondo punto]
        Dimostriamo che \( L(X) \) è un sottospazio vettoriale di \( V \).

        \textbf{(1) Non vuoto.}  
        Poiché \( L(X) \supseteq X \) e \( X \neq \emptyset \), segue immediatamente che \( L(X) \neq \emptyset \).

        \textbf{(2) Chiusura rispetto all’addizione.}  
        Siano \( u, u' \in L(X) \).  
        Allora, per definizione di chiusura lineare, esistono \( t, t' \in \mathbb{N} \), vettori \( u_1, \dots, u_t, u'_1, \dots, u'_{t'} \in X \) e scalari \( \alpha_1, \dots, \alpha_t, \alpha'_1, \dots, \alpha'_{t'} \in K \) tali che:
        \[
        u = \alpha_1 u_1 + \dots + \alpha_t u_t, \qquad
        u' = \alpha'_1 u'_1 + \dots + \alpha'_{t'} u'_{t'}.
        \]
        Sommando membro a membro otteniamo:
        \[
        u + u' = \alpha_1 u_1 + \dots + \alpha_t u_t + \alpha'_1 u'_1 + \dots + \alpha'_{t'} u'_{t'} \in L(X),
        \]
        poiché è ancora una combinazione lineare di vettori di \( X \).

        \textbf{(3) Chiusura rispetto alla moltiplicazione per scalare.}  
        Sia \( \lambda \in K \).  
        Allora:
        \[
        \lambda u = \lambda (\alpha_1 u_1 + \dots + \alpha_t u_t)
        = (\lambda \alpha_1)u_1 + \dots + (\lambda \alpha_t)u_t \in L(X),
        \]
        poiché anche in questo caso otteniamo una combinazione lineare di elementi di \( X \).  

        Pertanto, \( L(X) \) è chiuso rispetto alla somma e alla moltiplicazione per scalare, quindi è un sottospazio vettoriale di \( V \).
    \end{proof}

    \begin{proof}[Dimostrazione del terzo punto]
        Ipotizziamo che \( W \) sia un sottospazio vettoriale di \( V \) tale che \( X \subseteq W \).

        Vogliamo mostrare che \( L(X) \subseteq W \); cioè, per ogni \( u \in L(X) \), si ha \( u \in W \).

        Sia dunque \( u \in L(X) \). Per definizione di chiusura lineare, esistono \( u_1, \dots, u_t \in X \) e \( \alpha_1, \dots, \alpha_t \in K \) tali che:
        \[
        u = \alpha_1 u_1 + \dots + \alpha_t u_t.
        \]
        Poiché \( X \subseteq W \), ciascun \( u_i \in W \).  
        Poiché \( W \) è sottospazio vettoriale, è chiuso rispetto alla moltiplicazione per scalare:
        \[
        \alpha_i u_i \in W \quad \forall i.
        \]
        Inoltre, \( W \) è chiuso rispetto all’addizione, dunque:
        \[
        \alpha_1 u_1 + \dots + \alpha_t u_t \in W.
        \]
        Quindi \( u \in W \), e pertanto \( L(X) \subseteq W \).
    \end{proof}
}

\section{Dipendenza e indipendenza lineare}

\Definizione{Linearmente dipendente e indipendente}{
    Una \(n-pla (v_1,...,v_n)\) di vettori di \(V\) sarà detta \textbf{linearmente dipendente} se esiste
    una \(n-pla (\alpha_1,...,\alpha_n)\) di scalari non tutti nulli, tale che 
    \[\alpha_1 v_1 + ... + \alpha_n v_n = 0.\]
    In caso contrario, e cioè se
    \[(\alpha_1 v_1 + ... + \alpha_n v_n) \Rightarrow (a_i = 0, \forall i \in \mathbb{N}_n)\]
    la \(n-pla (v_1,...,v_n)\) sarà detta \textbf{linearmente indipendente}.
    \BoxBlue{Osservazione:}{}{
       Se \(X \subseteq T \subseteq V\), con \(X\) linearmente \textbf{dipendente} \(\Rightarrow T\) è linearmente \textbf{dipendente}. \newline
       Se \(Y \subseteq X \subseteq V\), con \(X\) linearmente \textbf{indipendente} \(\Rightarrow Y\) è linearmente \textbf{indipendente}.
    }
}

\Teorema{Caratterizzazione della dipendenza lineare}{
    Un sottoinsieme \(X\) di \(V\) è \textbf{linearmente dipendente} se e solo se 
    \[
    \exists\, u \in X \; \text{tale che} \; L(X) = L(X \setminus \{u\}).
    \]

    Non consideriamo il caso \(X = \emptyset\), poiché l’insieme vuoto è per definizione linearmente indipendente.  
    Quindi supponiamo \(X \neq \emptyset\).
    
    \medskip
    \textbf{Dimostrazione.}

    \textbf{(Direzione ``\(\Rightarrow\)''):}  
    Sia \(X\) linearmente dipendente.  
    Per definizione, esistono \(u_1, \dots, u_t \in X\) e scalari \(\alpha_1, \dots, \alpha_t \in K\), non tutti nulli, tali che:
    \[
    \vec{0} = \alpha_1 u_1 + \alpha_2 u_2 + \dots + \alpha_t u_t.
    \]
    Sia \(\alpha_t \neq 0\). Allora esiste \(\alpha_t^{-1} \in K\).  
    Moltiplicando entrambi i membri per \(\alpha_t^{-1}\), otteniamo:
    \[
    \vec{0} = (\alpha_t^{-1}\alpha_1)u_1 + \dots + (\alpha_t^{-1}\alpha_{t-1})u_{t-1} + (\alpha_t^{-1}\alpha_t)u_t.
    \]
    Ma \(\alpha_t^{-1}\alpha_t = 1\), quindi:
    \[
    u_t = -(\alpha_t^{-1}\alpha_1)u_1 - \dots - (\alpha_t^{-1}\alpha_{t-1})u_{t-1}.
    \]
    Da cui segue che \(u_t\) è combinazione lineare degli altri vettori \(u_1, \dots, u_{t-1}\), cioè:
    \[
    u_t \in L(\{u_1, \dots, u_{t-1}\}) \subseteq L(X \setminus \{u_t\}).
    \]
    Quindi:
    \[
    L(X) = L(X \setminus \{u_t\}),
    \]
    perché rimuovere \(u_t\) non cambia la chiusura lineare dell’insieme.  
    Ciò dimostra la direzione ``\(\Rightarrow\)``.

    \medskip
    \textbf{(Direzione ``\(\Leftarrow\)''):}  
    Supponiamo ora che esista \(u \in X\) tale che \(L(X) = L(X \setminus \{u\})\).  
    Allora \(u \in L(X \setminus \{u\})\), cioè:
    \[
    u = \alpha_1 u_1 + \dots + \alpha_t u_t,
    \]
    con \(u_1, \dots, u_t \in X \setminus \{u\}\) e \(\alpha_i \in K\).

    Portando tutto a primo membro otteniamo:
    \[
    \vec{0} = (-1)u + \alpha_1 u_1 + \dots + \alpha_t u_t.
    \]
    Questa è una combinazione lineare non banale (poiché il coefficiente di \(u\) è \(-1 \neq 0\)) che dà il vettore nullo.  
    Quindi \(X\) è linearmente dipendente.

    \medskip
    \(\Box\)
}




