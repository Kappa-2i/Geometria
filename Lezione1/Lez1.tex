\chapter{Lezione 1 - 24/09}

\section{Operazione Binaria}
\Definizione{Operazioni su Insiemi}{
    Siano \(A, B, C\) insiemi. Diremo \textbf{operazione binaria} ogni applicazione
    \[\varphi : A \times B \rightarrow C.\]
    \begin{itemize}
        \item Se in particolare \(A = B = C\), allora diremo che \(\varphi : A \times A \rightarrow A\) è
        un operazione binaria interna su \(A\).
        \item Se \(B = C\), allora diremo che \(\varphi\) si dice esterna con operatori in A.
    \end{itemize}
}

\section{Strutture Algebriche}

\BoxBlue{Ricorda:}{Strutture Algebriche}{
    \begin{itemize}
        \item \((S,\perp)\) si dice \textbf{semigruppo} se \(\perp\) è associativa;
        \item \((S,\perp)\) si dice \textbf{semigruppo commutativo} se \(\perp\) è semigruppo con commutatività;
        \item \((S,\perp)\) si dice \textbf{monoide} se \(\perp\) è semigruppo dotato di neutro;
        \item \((S,\perp)\) si dice \textbf{monoide commutativo} se \(\perp\) è monoide con commutatività;
        \item \((S,\perp)\) si dice \textbf{gruppo} se \(\perp\) è monoide dove ogni elemento è simmetrizzabile;
        \item \((S,\perp)\) si dice \textbf{gruppo abeliano} se \(\perp\) è gruppo con commutatività.
    \end{itemize}
}

\Definizione{Struttura Algebrica}{
    Per \textbf{struttura algebrica} si intende una \(n\)-upla costituita da insiemi e operazioni su di essi.
    La più semplice struttura algebrica, spesso detta \emph{gruppoide}, è una coppia
    \((X, \perp)\), dove \(X\) è un insieme e \(\perp\) è un operazione binaria interna su \(X\).
}








\subsection{Gruppo}

\Definizione{Gruppo}{
    Un \textbf{gruppo} è una struttura algebrica \((G, \perp)\) formata da un insieme \(G\) e da un'operazione binaria interna 
    \(\perp : G \times G \to G\), che soddisfa le seguenti proprietà:

    \begin{enumerate}
        \item \textbf{Associatività:} \(\forall\, a,b,c \in G,\; (a \perp b) \perp c = a \perp (b \perp c)\);
        \item \textbf{Elemento neutro:} esiste un elemento \(e \in G\) tale che \(\forall\, a \in G,\; a \perp e = e \perp a = a\);
        \item \textbf{Elemento inverso:} per ogni \(a \in G\) esiste un elemento \(\bar{a} \in G\) tale che 
            \(a \perp \bar{a} = \bar{a} \perp a = e\).
    \end{enumerate}

    Se inoltre vale la \textbf{proprietà commutativa}
    \[
        a \perp b = b \perp a \quad \forall\, a,b \in G,
    \]
    allora il gruppo si dice \textbf{abeliano}.
}

\Proposizione{Proprietà elementari in un gruppo}{
Sia \((X,\perp)\) un gruppoide (cioè \(X\) insieme non vuoto con un'operazione binaria \(\perp\) su \(X\)). Valgono le seguenti proprietà.

\begin{enumerate}
  \item Se \((X,\perp)\) ammette un elemento neutro, questo è unico.
  \item Se \((X,\perp)\) ammette un elemento neutro \(e\) e \(\perp\) è associativa, allora per ogni \(x\in X\) invertibile l'inverso di \(x\) è unico.
  \item Inoltre, se \(\perp\) è associativa e ha elemento neutro \(u \in A\) e abbiamo \(x, y \in A\) che hanno i loro inversi \(\bar{x}, \bar{y}\), allora \(x \perp y\) è invertibile e il suo inverso è \(\bar{y} \perp \bar{x}\).
  
\end{enumerate}

\textcolor{myred}{\textbf{Dimostrazioni:}}\\[0.5em]

\begin{proof}[Dimostrazione della proprietà 1 (unicità dell'elemento neutro)]
Supponiamo che \(e\) ed \(e'\) siano due elementi neutri in \((X,\perp)\). Per definizione di elemento neutro:
\[
\forall x\in X:\quad e\perp x = x \quad\text{e}\quad e'\perp x = x.
\]
In particolare, considerando \(x=e'\) nella prima uguaglianza e \(x=e\) nella seconda otteniamo
\[
e\perp e' = e' \quad\text{e}\quad e'\perp e = e.
\]
Se non assumiamo necessariamente commutatività, bperpa usare una delle due uguaglianze applicata all'altro neutro:
usando \(e\perp e' = e'\) e insieme \(e'\perp e = e\) otteniamo
\[
e = e'\quad(\text{poiché } e = e'\perp e = e').
\]
Quindi \(e = e'\) e l'elemento neutro è unico.
\end{proof}

\begin{proof}[Dimostrazione della proprietà 2 (unicità dell'inverso e formula dell'inverso del prodotto)]
Sia \((X,\perp)\) associativo e con elemento neutro \(e\). 

\emph{Unicità dell'inverso.} Sia \(x\in X\) e supponiamo che \(y\) e \(y'\) siano due inversi di \(x\), cioè
\[
x\perp y = e = y\perp x,\qquad x\perp y' = e = y'\perp x.
\]
Allora, usando l'associatività,
\[
y = y\perp e = y\perp (x\perp y') = (y\perp x)\perp y' = e\perp y' = y'.
\]
Quindi \(y=y'\) e l'inverso è unico.
\end{proof}

\begin{proof}[Dimostrazione della proprietà 3]
    Siano \((A,\perp)\) un insieme con operazione binaria \(\perp\), associativa, e dotato di elemento neutro \(u\in A\).
    Supponiamo \(x,y\in A\) invertibili e indichiamo con \(\overline{x}\) e \(\overline{y}\) i loro inversi, cioè
    \[
    x\perp\overline{x}=\overline{x}\perp x = u,\qquad
    y\perp\overline{y}=\overline{y}\perp y = u.
    \]
    
    Consideriamo il candidato \( \overline{y}\perp\overline{x} \) come possibile inverso di \(x\perp y\).
    Calcoliamo il prodotto a destra:
    \[
    \begin{aligned}
    (x\perp y)\perp(\overline{y}\perp\overline{x})
    &\stackrel{\text{(assoc.)}}{=}
    x\perp\bigl(y\perp(\overline{y}\perp\overline{x})\bigr) \\
    &= x\perp\bigl((y\perp\overline{y})\perp\overline{x}\bigr) \\
    &= x\perp(u\perp\overline{x}) \\
    &= x\perp\overline{x} \\
    &= u.
    \end{aligned}
    \]
    
    Analogamente, il prodotto a sinistra:
    \[
    \begin{aligned}
    (\overline{y}\perp\overline{x})\perp(x\perp y)
    &\stackrel{\text{(assoc.)}}{=}
    \overline{y}\perp\bigl(\overline{x}\perp(x\perp y)\bigr) \\
    &= \overline{y}\perp\bigl((\overline{x}\perp x)\perp y\bigr) \\
    &= \overline{y}\perp(u\perp y) \\
    &= \overline{y}\perp y \\
    &= u.
    \end{aligned}
    \]
    
    Quindi \(\overline{y}\perp\overline{x}\) è sia inverso a destra sia inverso a sinistra di \(x\perp y\). 
    Poiché in una struttura con elemento neutro e operazione associativa l'inverso (se esiste) è unico, segue che
    \[
    \overline{(x\perp y)} = \overline{y}\perp\overline{x},
    \]
    come volevamo dimostrare.
\end{proof}
    
    
}




\subsection{Anello}

\Definizione{Anello}{
  Un \textbf{anello} è una struttura algebrica \((A, +, \cdot)\) formata da un insieme \(A\) e da due operazioni binarie interne:
  \[
    + : A \times A \to A, \qquad \cdot : A \times A \to A,
  \]
  tali che valgono le seguenti proprietà:

  \begin{enumerate}
    \item \((A, +)\) è un \textbf{gruppo abeliano}
    \item L’operazione \(\cdot\) (detta \emph{moltiplicazione}) è associativa
    \item La moltiplicazione è distributiva rispetto all’addizione:
      \[
        a \cdot (b + c) = a \cdot b + a \cdot c, \quad 
        (a + b) \cdot c = a \cdot c + b \cdot c, \quad \forall\, a,b,c \in A.
      \]
  \end{enumerate}

  \vspace{0.5em}
  Se esiste un elemento \(1 \in A\) tale che \(1 \cdot a = a \cdot 1 = a\) per ogni \(a \in A\), 
  l’anello si dice \textbf{unitario} o \textbf{con elemento neutro moltiplicativo}.  

  Se inoltre la moltiplicazione è commutativa, l’anello si dice \textbf{commutativo}.
}







\subsection{Campo}
\Definizione{Campo}{
  Un \textbf{campo} è una struttura algebrica \((K, +, \cdot)\) formata da un insieme \(K\) e da due operazioni binarie interne:
  \[
    + : K \times K \to K, \qquad \cdot : K \times K \to K,
  \]
  che soddisfano le seguenti proprietà:

  \begin{enumerate}
    \item \((K, +)\) è un \textbf{gruppo abeliano}:
    \item \((K \setminus \{0\}, \cdot)\) è un \textbf{gruppo abeliano} rispetto alla moltiplicazione:
    \item Le due operazioni sono collegate dalle proprietà distributive:
  \end{enumerate}

  \vspace{0.5em}
  In altre parole, un campo è un \textbf{anello commutativo con elemento unità} in cui \textbf{ogni elemento non nullo è invertibile}.
}







\subsection{Spazio Vettoriale}

\Definizione{Spazio vettoriale}{
Diremo che \((V,\boxplus,\boxdot)\) è \textbf{spazio vettoriale} sul campo \(\mathbb{R}\) se:
\begin{enumerate}
    \item \(V \neq \emptyset\) \newline
          Operazione interna: \(\boxplus  : V \times V \to V\), \((u, v) \mapsto u + v\) \newline
          Operazione esterna: \(\boxdot : K \times V \to V\), \((\lambda, v) \mapsto \lambda v\)
    \item \((V,\boxplus)\) è gruppo abeliano, \(\forall \underline{v}, \underline{w}, \underline{z} \in V\)
        \begin{enumerate}
            \item Proprietà associativa: \((\underline{v} \boxplus \underline{w} ) \boxplus \underline{z} = \underline{v} \boxplus (\underline{w} \boxplus \underline{z})\)
            \item Proprietà commutativa: \(\underline{v} \boxplus \underline{w} = \underline{w} \boxplus \underline{v}\)
            \item Esiste opposto: \(\underline{v} \boxplus (-\underline{v}) = \underline{0}\)
            \item Esiste elemento neutro: \(\underline{v} \boxplus \underline{0} = \underline{0} \boxplus \underline{v} = \underline{v}\)
        \end{enumerate}
    \item Esiste elemento neutro rispetto a \(\boxdot\): \(1 \boxdot \underline{v} = \underline{v} = \underline{v} \boxdot 1\)
    \item \textbf{Per tutti } \(h,k \in \mathbb{R}, \underline{v}, \underline{w} \in V\) valgono le seguenti proprietà:
        \begin{enumerate}
            \item Compatibilità della moltiplicazione scalare: \((h \cdot k) \boxdot \underline{v} = h \boxdot (k \boxdot \underline{v})\)
            \item Distributività rispetto alla somma scalare: \((h + k) \boxdot \underline{v} = (h \boxdot \underline{v}) \boxplus (k \boxdot \underline{v})\)
            \item Distributività rispetto alla somma vettoriale: \(h \boxdot (\underline{v} \boxplus \underline{w}) = (h \boxdot \underline{v}) \boxplus (h \boxdot \underline{w})\)
        \end{enumerate}
\end{enumerate}
}

\Proposizione{Proprietà aritmetiche degli Spazi Vettoriali}{
    Sia \(V\) uno spazio vettoriale sul campo \(\mathbb{K}. \forall \alpha, \beta \in \mathbb{K}, \forall v \in V\) si ha:
    \begin{enumerate}
        \item \(\alpha \boxdot v = 0 \iff \alpha = 0\) oppure \(v=0\)
        \item \((-\alpha)\boxdot v = \alpha \boxdot (-v) = -(\alpha \boxdot v)\)
        \item se \(\alpha \boxdot v = \beta \boxdot v \) e \(v \neq \varnothing\), allora \(\alpha = \beta\) 
        \item se \(\alpha \boxdot u = \alpha \boxdot v \) e \(\alpha \neq \varnothing\), allora \(u = v\)
    \end{enumerate}

    \begin{proof}[Dimostrazione delle proprietà]
    
        \textbf{(i)} Dimostriamo che 
        \[
            a v = 0 \iff a = 0 \text{ oppure } v = 0.
        \]
        \textbf{Direzione ``$\Rightarrow$''}: se \(a = 0\) oppure \(v = 0\), allora chiaramente \(a v = 0\). Infatti:
        \begin{itemize}
            \item se \(a = 0\), per le proprietà dello spazio vettoriale abbiamo \(0 \cdot v = 0\);
            \item se \(v = 0\), allora \(a \cdot 0 = 0\).
        \end{itemize}
    
        \textbf{Direzione ``$\Leftarrow$''}: supponiamo che \(a v = 0\) e \(a \neq 0\).  
        Poiché \(a \neq 0\), esiste l'inverso \(a^{-1} \in K\). Moltiplicando entrambi i membri per \(a^{-1}\) otteniamo:
        \[
            a^{-1}(a v) = a^{-1} \cdot 0 = 0.
        \]
        Usando l'associatività della moltiplicazione scalare:
        \[
            (a^{-1} a)v = 1 \cdot v = v = 0.
        \]
        Quindi, se \(a \neq 0\) e \(a v = 0\), necessariamente \(v = 0\).  
        Pertanto, la proprietà (i) è dimostrata.
    
        \medskip
        \textbf{(ii)} Dimostriamo ora che
        \[
            (-a)v = a(-v) = -(a v).
        \]
        Si ha immediatamente:
        \[
            a v + (-a)v = (a + (-a))v = 0v = 0,
        \]
        e anche
        \[
            a v + a(-v) = a(v + (-v)) = a0 = 0.
        \]
        Poiché in entrambi i casi la somma è nulla, segue che
        \[
            (-a)v = - (a v) = a(-v).
        \]
    
        \medskip
        \textbf{(iii)} Se \(a v = \beta v\), allora
        \[
            (a + (-\beta))v = a v + (-\beta)v = \beta v + (-\beta)v = 0.
        \]
        Poiché \(v \neq 0\) per ipotesi, per la proprietà (i) segue che \(a + (-\beta) = 0\), e quindi \(a = \beta\).
    
        \medskip
        \textbf{(iv)} Se \(a u = a v\) e \(a \neq 0\), allora
        \[
            a(u + (-v)) = a u + a(-v) = a u - a v = 0.
        \]
        Poiché \(a \neq 0\), per la proprietà (i) si ha \(u + (-v) = 0\), cioè \(u = v\).
    
    \end{proof}
    
        
}
